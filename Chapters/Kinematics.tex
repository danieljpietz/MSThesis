\chapter{DYNAMICS} \label{chap:Kinematics}
Forward kinematics is the process of computing the motion of points on a robot, typically an end-effector \Cite{paul_1992} from a specified set of joint parameters. In the case of OKC's this is done iteratively beginning with the root frame of the system and branching outward. This gives the following
\begin{align}
	\begin{split}                                                           
	{^I\T_B} = {^I\T_{B-1}}{^{B - 1}\T_{B}}                                 
	\end{split}                                                             
	\begin{split}                                                           
	{^I_I\r_B} =  {^I_I\r_{B - 1}} + {^{I}\T_{B-1}}{^{B - 1}_{B - 1}\r_{B}} 
	\end{split}                                                             
\end{align}

\noindent The linear and angular velocities of a frame are found by computing its Geometric Jacobian matrix, $\J_B$. This matrix, maps the joint velocities of the system to the spatial velocities of $B$ \Cite{shabana_2010}. Similarly, the time derivative of this matrix ($\dotJ_B$) can be used to solve for accelerations.
\begin{align*}
	\begin{split}                               
	\begin{bmatrix}                             
	{\BB\omega_{I}}                             \\
	{^I_I\dot{r}_B}                             
	\end{bmatrix} = \J_B \dotx                  
	\end{split}                                 
	\begin{split}                               
	\begin{bmatrix}                             
	{\BB\dot{\omega}_{B - 1}}                  \\
	{^I_I\ddot{r}_B}                            
	\end{bmatrix} = \dotJ_B \dotx + \J_B \ddotx 
	\end{split}                                 
\end{align*}

These matrices are also useful for projecting real-world mass of a body into the joint space. Let $\M_B$ denote the \emph{spatial inertia matrix} of a link $B$, defined as \Cite{featherstone_2007}:
\begin{align} \label{eqn:spat_mass}
	\M_B =
	\begin{bmatrix}
	\BB\J                 & \S{\BB\com}{^B\T_I} \\
	{^I\T_B}\S{\BB\com}^T & m_B \EYE_{3x3}      
	\end{bmatrix}
\end{align}

\noindent From this, the system mass matrix and vector of fictitious forces vector described in \ref{eqn:NE1} can be computed \Cite{isenberg_2020}.
\begin{align} \label{eqn:NE2}
	\begin{split}
	\H    & = \Big.\sum_B \H_B                           \\
	\d    & = \Big.\sum_B \d_B                           \\
	\Fn   & =  \Big.\sum_B \Fn_B                         
	\end{split}
	\begin{split}
	\H_B  & = {\J^T_B}\M_B\J_B                           \\
	\d_B  & = \J^T_B \left (\M_B \dotJ_B + \d'_B \right) \\
	\d_B' & = \begin{bmatrix}                            
	{\BB\omega_I } \times {\BB\J} {\BB\omega_I} \\
	{^I\T_B}  (\BB\omega_I \times (\BB\omega_I \times \com))\\
	\end{bmatrix} \\
	\Fn_B & = \F_{B_J} + \J_B^T \F_{B_W}
	\end{split} 
\end{align}

%TODO: Workshop this sentence
\noindent While this gives clear application to {$\J_B$} and $\dotJ_B$, it does not give a way to compute them. Like the systems equations of motion, the symbolic definitions for these matrices grow rapidly with the number of degrees of freedom. To avoid this, techniques were developed to compute these terms numerically, such as that developed by \Cite{isenberg_2020}.

\section{Recursive Jacobian Computation}
\noindent When analyzing OKC systems, the motion of a link provides a point of reference for all links in its subtree. A recursive definition for the Jacobian can be developed, starting from the root and iterating outward. %From \Cite{isenberg_2020}, we get Lemma \ref{lem:recurse_jacobian}.

\begin{lemma} \label{lem:recurse_jacobian}
	Given a body $B$ with parent $B - 1$, the geometric Jacobian of body $B$,
	\begin{align*}
		\J_B & = \J_B' \J_{B-1} + \begin{bmatrix} 
		{\IHat_B} \\
		{{^I\T_{B-1}}\ITilde_B}
		\end{bmatrix}
	\end{align*}
	\noindent where 
	\begin{align*}
		\begin{split}
		\J_B' = \begin{bmatrix}
		{^B\T_{B - 1}}                                       & \ZERO_{3 \times 3} \\
		\left (\S{^{B-1}_{B-1}\r_B}{^{B-1}\T_{I}} \right )^T & \EYE_{3 \times 3}  
		\end{bmatrix}
		\end{split}
		\begin{split}
		\begin{bmatrix}
		{\BB\omega}_{B-1} \\
		{^{B-1}_{B-1}}\dotr_B 
		\end{bmatrix} = 
		\begin{bmatrix}
		\IHat_B\\
		\ITilde_B\\
		\end{bmatrix}  \dotx
		\end{split}
	\end{align*}
	\begin{proof}
		\begin{align*}
			\J_B \dotx                                           & = \begin{bmatrix}  
			{\BB\omega_I} \\
			{^I_I\dotr_B}
			\end{bmatrix} \\
			                                                     & = \begin{bmatrix}  
			{{_{B - 1}^{~~~B}\omega_{I}}} \\
			{^{I}_{I}\dotr_{B - 1}}
			\end{bmatrix} + \begin{bmatrix}
			{\BB\omega_{B - 1}} \\
			{{^I\T_{B-1}}^{B - 1}_{B - 1}\dotr_B}
			\end{bmatrix} \\
			                                                     & = \begin{bmatrix}  
			{{{^B}\T_{B-1}}{^{B - 1}_{B - 1}\omega_{I}}} \\
			{^{I}_{I}\dotr_{B - 1} + {^I\T_{B - 1} \left ( ^{B-1}_{B-1}\omega_I \times ^{B-1}_{B-1}\r_B \right )}}
			\end{bmatrix} + \begin{bmatrix}
			{\BB\omega_{B - 1}} \\
			{{^I\T_{B-1}}^{B - 1}_{B - 1}\dotr_B}
			\end{bmatrix} \\
			                                                     & =                  
			\begin{bmatrix}
			{^B\T_I}                                             & \ZERO_{3 \times 3} \\
			\left (\S{^{B-1}_{B-1}\r_B}{^{B-1}\T_{I}} \right )^T & \EYE_{3 \times 3}  
			\end{bmatrix}
			\begin{bmatrix}
			{^{B - 1}_{B - 1}\omega_{I}} \\
			{^{I}_{I}\dotr_{B - 1}}
			\end{bmatrix} + \begin{bmatrix}
			{\BB\omega_{B - 1}} \\
			{{^I\T_{B-1}}^{B - 1}_{B - 1}\dotr_B}
			\end{bmatrix} \\
			                                                     & =                  
			\J_B'\J_{B-1}\dotx + \begin{bmatrix}
			{\IHat_B} \\
			{{^I\T_{B-1}}\ITilde_B}
			\end{bmatrix} \dotx \\
			                                                     & =                  
			\left ( \J_B'\J_{B-1} + \begin{bmatrix}
			{\IHat_B} \\
			{{^I\T_{B-1}}\ITilde_B}
			\end{bmatrix} \right) \dotx 
		\end{align*}
		\noindent Therefore
		\begin{align} \label{eqn:recurse_jacob}
			\J_B & = \J_B' \J_{B-1} + \begin{bmatrix} 
			{\IHat_B} \\
			{{^I\T_{B-1}}\ITilde_B} \\
			\end{bmatrix}  
		\end{align}
	\end{proof}
\end{lemma}
\noindent Differentiating equation \ref{eqn:recurse_jacob} yields the recursive expression for $\dotJ_B$
\begin{align}
	\dotJ_B & = \dotJ_B' \J_{B-1} + \J_B' \dotJ_{B-1} + \begin{bmatrix} 
	\ZERO_{3 \times n} \\
	{{^I\dot{\T}_{B-1}}\ITilde_B} \\
	\end{bmatrix} \nonumber  \\
	        & = \dotJ_B' \J_{B-1} + \J_B' \dotJ_{B-1} + \begin{bmatrix} 
	\ZERO_{3 \times n} \\
	{{^I{\T}_{B-1}}\S{^{B-1}_{B-1}\omega_I}\ITilde_B} \\
	\end{bmatrix}  
\end{align}
\noindent where
\begin{align} \label{eqn:recurse_dotjacob}
	\dotJ_B'                                                               & = \begin{bmatrix}  
	-\S{\BB\omega_{B-1}}{^B\T_{B-1}}                                       & \ZERO_{3 \times 3} \\
	-{^I\T_{B-1}}
	(\S{^{B-1}_{B-1}\omega_I}\S{^{B-1}_{B-1}\r_B}+\S{^{B-1}_{B-1}\dotr_B)} & \ZERO_{3 \times 3} \\
	\end{bmatrix}
\end{align}

\noindent Substituting equations \ref{eqn:recurse_jacob} - \ref{eqn:recurse_dotjacob} into \ref{eqn:NE2} allows for the joint accelerations in  \ref{eqn:NE1} to be solved. These accelerations, alongside joint velocities, give $f$ in equation \ref{eqn:lie_expanded}.

\section{Forces} \label{sec:forces}

\noindent The forces acting on an object will vary depending on the system being modeled. While it is impossible to generalize all types of forces that could act on a body, general cases can be derived common forces that show up often. These are split into two main categories, Joint Space and World Space. 

\subsubsection{Joint Space Forces}
Joint Space forces act directly on the axis of motion of a link, and includes forces such as actuation friction, actuation limits, joint springs and dampeners, etc. A joint space force is a vector $\hat{\mathbf{k}} \in \mathbb{R}^n$, with indices in $\hat{\mathbf{k}}$ corresponding to the forces on individual links. Multiple forces in this space can be added together to compute $\F_{B_J}$. Given a set of forces $\mathbf{\hat{K}_B} = \left [ \mathbf{\hat{k}_1} \quad \mathbf{\hat{k}_2} \quad \hdots \right ]$ acting on body $B$, 
\begin{align}
	\F_{B_J} = \Big.\sum_{\mathbf{\hat{k}_i} \in \mathbf{\hat{K}_B}}\mathbf{\hat{k}_i}
\end{align}

\noindent Forces such as friction are trivial to compute in this space \Cite{lynch_park_2019}. For example, modeling Coloumbic friction on a body $B$ gives the force vector 
\begin{align} \label{eqn:coulomb}
    \mathbf{\hat{k}_i} = 
    \begin{cases}
         - \mu_{\mathbf{\hat{k}_i}}\text{sign}(\dotx_i) & i = B \\
         0 & \text{otherwise}
    \end{cases}
\end{align}

\noindent For viscous friction,
\begin{align} \label{eqn:viscous}
    \mathbf{\hat{k}_i} = 
    \begin{cases}
         - \mu_{\mathbf{\hat{k}_i}}\dotx_i & i = B \\
         0 & \text{otherwise}
    \end{cases}
\end{align}

\noindent For linear springs,
\begin{align} \label{eqn:spring}
    \mathbf{\hat{k}_i} = 
    \begin{cases}
         - k_{\mathbf{\hat{k}_i}}\Delta \x_i & i = B \\
         0 & \text{otherwise}
    \end{cases}
\end{align}

\subsubsection{World Space Forces}

World space forces are external forces acting at a point on the rigid body, such as gravity and contact forces. A world space force is the 2-tuple $\mathbf{\hat{v}} = ({^I\hat{f}}_{\mathbf{\hat{v}}} \in \mathbb{R}^3, {\BB}\hat{r}_{\hat{\mathbf{\hat{v}}}} \in \mathbb{R}^3)$ corresponding to the force vector and position of $\mathbf{\hat{v}}$. \newline

\noindent The torque placed on a body $B$ from a force $\mathbf{\hat{v}}$ can be computed by taking the cross product between the position and direction of the first relative to $B$. As such, the Spatial Force Vector $\mathbf{\hat{V}}$ is
\begin{align} \label{eqn:spat_force}
	\mathbf{\hat{V}} = \begin{bmatrix}                                                    
	{{\BB}}\hat{r}_{\hat{\mathbf{\hat{v}}}} \times {^B\T_I}{^I\hat{f}}_{\mathbf{\hat{v}}} \\ {^I\hat{f}}_{\mathbf{\hat{v}}} 
	\end{bmatrix}                                                                     
\end{align}

\noindent Per this definition, $\mathbf{\hat{V}}$ is the torque and force felt by $B$ measured in the inertial frame. Because this is measured in the inertial frame, the SFV for all forces acting on $B$ can be summed together to get the total torque and force. This is then transformed into the joint space. Given a set of forces $\mathbf{\hat{V}_B} = \left [ \mathbf{\hat{v}_1} \quad \mathbf{\hat{v}_2} \quad \hdots \right ]$ acting on body $B$, 
\begin{align}
	\F_{B_W} = \Big.\sum_{\mathbf{\hat{v}_i} \in \mathbf{\hat{V}_B}}\begin{bmatrix}            
	{{\BB}}\hat{r}_{\hat{\mathbf{\hat{v}}}} \times {^B\T_I}{^I\hat{f}}_{\mathbf{\hat{v}}}          \\ {^I\hat{f}}_{\mathbf{\hat{v}}} 
	\end{bmatrix} =  \Big.\sum_{\mathbf{\hat{v}_i} \in \mathbf{\hat{V}_B}}\mathbf{\hat{V}_i} 
\end{align}

\noindent This force definition can be used to derive general expressions for common forces, such as gravity.

\begin{lemma}
	Given a gravitational acceleration vector $^I\hat{G}$, acting on body $B$, the spatial force vector 
	\begin{align*}
		\mathbf{\hat{G}} = m_B\begin{bmatrix} 
		\S{\BB\com_B}{^B\T_I}{^I\hat{G}}      \\ {^I\hat{G}}
		\end{bmatrix}                         
	\end{align*}
	\begin{proof}
		\noindent A common convention used to represent gravity is a force acting at the center of mass of the body. The strength of this force is equal to the product of mass and acceleration vectors. Thus, our force $\mathbf{\hat{g}} = (m_B^I\hat{G},  {\BB}\com)$. Plugging $\mathbf{\hat{g}}$ into equation \ref{eqn:spat_force} yields
		\begin{align*}
			\mathbf{\hat{G}} = \begin{bmatrix}      
			{{\BB}}\com \times {^B\T_I}m_B^I\hat{G} \\ m_B^I\hat{G} 
			\end{bmatrix} = m_B\begin{bmatrix}      
			\S{\BB\com_B}{^B\T_I}{^I\hat{G}}        \\ {^I\hat{G}}
			\end{bmatrix}                           
		\end{align*}
	\end{proof}
\end{lemma}